\documentclass[12pt]{article}
 
\usepackage[margin=1in]{geometry} 
\usepackage[utf8]{inputenc}
\usepackage{polski} 
\usepackage{hyperref}

\title{%
Elektroniczna szachownica z komputerem szachowym \\
\large Inspirowana urządzeniem \href{https://www.dgt.nl/index.php/products/chess-computers/dgt-centaur}{DGT Centaur}
}
\author{Gabriel Wechta (250111),\\ Patryk Majewski (250134) }
\date{}

\begin{document}

\maketitle

\section{Wymagania systemu}
\subsection{Co robi system?}
\begin{enumerate}
    \item Pozwala na rozegranie partii szachowej w tradycyjnej formie, to jest fizyczna szachownica i bierki. Umożliwia grę z komputerem lub człowiekiem, oferując dodatkowo ewaluację ruchów. W przypadku potyczki z komputerem, użytkownik ma możliwość wybrania spośród trzech poziomów trudności. Może też zdecydować, jakim kolorem będzie grać komputer. Gra kończy się w jednym z następujących przypadków: szach mat, pat, remis.
    \item Podobnie jak tradycyjna szachownica, system jest przenośny i nie wymaga stałego zasilania.
    \item Umożliwia promocję piona z wyborem figury, w którą ma się zmienić.
    \item Obsługuje roszady, które gracz wprowadza za pomocą odpowiednich sekwencji ruchów.
    \item Na ekranie wyświetla obecny stan gry, zegar szachowy, życie baterii, ewaluację obecnej pozycji, historię ruchów i podpowiedź następnego ruchu.
    \item Komunikuje się z użytkownikiem poprzez podświetlanie poszczególnych pól. Sygnalizuje zaakceptowanie ruchu poprzez podświetlenie pola, na które gracz przesunął bierkę. W przypadku gry z komputerem, szachownica sygnalizuje jego ruch poprzez zaznaczenie pola z bierką, którą należy przestawić, jak również pola docelowego.
    \item Kontroluje zgodność ruchów z zasadami gry. W przypadku zagrania ruchu niezgodnego z zasadami, na ekranie pojawia się komunikat o nielegalnym ruchu. Obecne i źródłowe pole bierki zaczyna mrugać.
    \item Umożliwia rozpoczęcie gry z dowolnej pozycji. 
    \item Pozwala na zapamiętanie ułożenia bierek, aby kontynuować przerwaną grę po ponownym włączeniu urządzenia. Wówczas na ekranie wyświetlany jest stan gry przed jej przerwaniem.
    \item Umożliwia rozgrywkę z ograniczeniem czasowym lub bez niego. Użytkownik może wybrać jeden z predefiniowanych czasów dla obu graczy. W każdym momencie ma możliwość wstrzymania gry. Zegar odmierzający czas gracza zatrzymuje się w momencie postawienia przez niego bierki na polu docelowym.
    \item Prowadzi ewaluację pozycji na szachownicy. Oblicza najlepsze ruchy w danym momencie. W przypadku rozgrywki z komputerem, wybiera ruchy odpowiednie dla wybranego poziomu trudności. W przypadku wykonania w imieniu komputera ruchu innego niż ten przez niego wskazany, silnik adaptuje się do obecnej sytuacji i nie zgłasza błędu.
    \item Na życzenie użytkownika system wyświetla na ekranie najkorzystniejszy dla niego w danym momencie ruch.
    \item W przypadku ruchu, którego nie jest w stanie odpowiednio zarejestrować, system informuje na ekranie o konieczności cofnięcia ruchu do ostatniej zarejestrowanej, legalnej pozycji.
    \item W przypadku przypadkowego przewrócenia bierek, system to wykrywa i sygnalizuje potrzebę powrotu do ostatniej zarejestrowanej legalnej pozycji.
    \item System po 30 minutach bez ruchu, w celu oszczędzania energii, wyłącza się.
    \item Aby wyłączyć urządzenie należy wcisnąć i przytrzymać przycisk Play/Pause - On/Off na 10 sekund.
\end{enumerate}

\subsection{Jakie jest środowisko?}
\begin{enumerate}
\item Urządzenie przeznaczone jest do użytku w pomieszczeniach. Powinno być przystosowane przynajmniej do warunków nieekstremalnych, to jest temperatura 0-35 °C, wilgotność powietrza 30-60\%, wysokość bezwzględna do 3000 m. 
\item Urządzenie ma być zamiennikiem dla tradycyjnej szachownicy. Musi być lekkie i łatwe w transporcie. Nie może wymagać stałego zasilania.
\end{enumerate}

\subsection{Kto korzysta z systemu i w jaki sposób?}
\begin{enumerate}

\item Użytkownik może korzystać z systemu na dwa sposoby:
\begin{enumerate}
    \item Grając przeciwko komputerowi, co wymaga realizowania przez system funkcjonalności związanych z oznajmieniem ruchów komputera i fizycznego przestawiania przez użytkownika bierek komputera i jego własnych,
	\item Grając przeciwko drugiemu graczowi, wtedy system zapisuje i wyświetla wykonane ruchy oraz przeprowadza ewaluację punktową obecnego stanu gry.
\end{enumerate}

\item Z innego punktu widzenia, użytkowników można podzielić według poziomu doświadczenia i oczekiwania co do nauki płynącej z rozgrywanych partii:
\begin{enumerate}
\item Gracz rozpoczynający naukę chce w głównej mierze dowiedzieć się, czy wykonany przez niego ruch był dobry i czy istnieje lepszy -- a jeżeli tak, jaki. Potrzebuje poziomu trudności pozwalającego na cieszenie się z gry,
\item Gracz doświadczony oczekuje satysfakcjonującego wyzwania i godnego, responsywnego przeciwnika, być może z informacją zwrotną co do jakości wykonywanych ruchów.
\end{enumerate}

\end{enumerate}

\subsection{W jaki sposób system realizuje założenia?}
\begin{enumerate}
\item Urządzenie do rozpoznania bierki na polu używa czujników ciężaru, których minimalna średnica detekcji wynosi 19 mm. Rodzaj i kolor bierki jest rozpoznawany na podstawie jej wagi. Tym samym przesunięcie bierki jest rejestrowany na podstawie zmiany masy.
\item System wyświetla i zapisuje rozgrywkę w formacie PGN.
\item Zbicie figury jest realizowane poprzez zarejestrowanie usunięcia ciężaru z pola, na którym figura się znajdowała, i położenie w jej miejsce bijącej figury.
\item Sytuacja na szachownicy jest zapisywana w pamięci urządzenia. Po zakończeniu rozgrywki historia ruchów jest zapomniana, robiąc miejsce na kolejną grę.
\item Ewaluacja sytuacji na szachownicy jest realizowana przez silnik szachowy Centaur zainstalowany na urządzeniu.
\item Silnik na podstawie swoich algorytmów wykonuje ewaluacje co ruch, dzięki czemu może adaptować się do posunięć wykonanych w jego imieniu, ale innych niż zgłosił.
\item Trudność wybrana przez gracza przed rozpoczęciem rozgrywki jest realizowana poprzez wybór ruchów o jakości odpowiedniej do poziomu trudności. 
\item Promocja piona jest realizowana poprzez zdjęcie piona po dojściu na linię przemiany, oraz położeniu w jego miejscu figury.
\item System rozpoznaje nielegalny ruch, sprawdzając legalność ruchu przez zaprogramowane zasady gry w szachy.
\item Wyświetlanie podpowiedzi realizowane jest poprzez wyświetlenie na ekranie najlepszego ruchu obliczonego przez silnik.
\item System wyświetla i odlicza czas każdemu z graczy. W przypadku naciśnięcia przycisku Play/Pause - On/Off, zegar jest zatrzymywany do czasu następnego wciśnięciu przycisku.
\item Odmierzanie czasu jest realizowane przez moduł RTC.
\item Urządzenie wyposażone jest w baterię litowo-polimerową, przystosowaną do urządzeń przenośnych.
\item Szachownica, wyświetlacz i czujniki ciężaru są zasilane przez baterię.
\item Bateria jest ładowana poprzez podłączenie urządzenia do ładowarki. Ładowarka USB - 5V i 2000mA.
\item Wyświetlacz jest wykonany w technologii e-Ink, dzięki czemu ma niskie zapotrzebowanie na energię.
\item Elementy świetlne szachownicy to diody LED o niskim zapotrzebowaniu na energię. Diody służą do informacji zwrotnej o zarejestrowaniu ruchu oraz sygnalizują ruch wykonany przez komputer. Urządzenie pozwala na dostosowanie intensywności światła, co umożliwia grę w różnych warunkach oświetleniowych.
\item Wymienione komponenty są powszechnie używane w urządzeniach przenośnych i mają wymagania środowiskowe mniej restrykcyjne od wcześniej wspomnianych.
\item Urządzenie jest zbliżone rozmiarem do tradycyjnej szachownicy (50 cm x 50 cm) i wykonane z tworzywa sztucznego, jest zatem trwałe i łatwe w transporcie.

\item Ustawienia urządzenie obsługiwane są przez następujące przyciski znajdujące się na obudowie:
\begin{itemize}
    \item Up -- przesunięcie wybranego pola w menu do góry.
    \item Down -- przesunięcie wybranego pola w menu w dół.
    \item Back -- powrót do poprzedniego ekranu, w przypadku przytrzymania wyświetlacz jest odświeżany. 
    \item Ok/Menu -- otwarcie ekranu ustawień, potwierdzenie wyboru.
    \item Hint/Alternate move -- pierwsze wciśnięcie: wyświetlenie najlepszego ruchu w pozycji, ponowne wciśnięcie: wyświetlenie alternatywy.
    \item Play/Pause - On/Off -- włącz/wyłącz. Gdy gra toczy się z włączonym zegarem, zatrzymaj/wznów czas.
\end{itemize}

\item Menu główne zawiera interfejs obsługujący: tryby gry, zegar szachowy, funkcjonalność rozpoczęcia rozgrywki od ustawionej pozycji, ustawienia jasności wyświetlacza, ustawienia wielkości czcionki na wyświetlaczu.
\item Aby móc się wyłączyć po określonym czasie, system odmierza czas od ostatniego ruchu używając modułu RTC.
\item Przy wyłączeniu z dowolnego powodu, obecny stan gry jest zapisywany w pamięci urządzenia, używając formatu PGN. Urządzenie odcina dopływ prądu do wyświetlacza.
\end{enumerate}

\end{document}
