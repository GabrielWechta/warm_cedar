\documentclass{article}
\usepackage{amsmath}
\usepackage[utf8]{inputenc} % `utf8` option to match Editor encoding
\usepackage[T1]{fontenc}
\usepackage[ruled,vlined]{algorithm2e}
\usepackage{tikz}
\usetikzlibrary{trees}

\begin{document}
\section{Lista 3, Zadanie 9}
Zadanie polega na przeanalizowaniu i określeniu całkowitej złożoności algorytmów rekurencyjnych, dla których określony jest podział na ilość pod-problemów, rozmiar pod-problemu i koszt scalania.

\subsection{Algorytm A}
$T(n) = 5T(\frac{n}{2}) + O(nlog(n))$\\

Jako, że mamy bardzo elegancko zadane współczynniki możemy skorzystać \textit{Twierdzenia o Rekurencji Uniwersalnej}.
U nas: 
\begin{align*}
	&a = 5\\
	&b = 2\\
	&\log_2 5 \approx 2.32
\end{align*}
$d$ z calą pewnością jest mniejsze od $2.32$ zatem $T(n) = O(n^{\log_2 5})$

\subsection{Algorytm B}
$T(n) = 2T(n-1)+ O(n)$\\

W tym przypadku nie jest już tak fajnie. Algortym nie dzieli problemu na problemy k-razy mniejsze a na o 1 mniejsze. Przez to nie możemy skorzystać z \textit{Twierdzenia o Rekurencji Uniwersalnej}.\\
Przeanalizujmy koszt algorytmu na kolejnych poziomach drzewa binarnego. (Wartości więzłów są kosztami).\\

\begin{tikzpicture}[level distance=1.5cm,
  level 1/.style={sibling distance=4.5cm},
  level 2/.style={sibling distance=3cm},
  level 3/.style={sibling distance=1cm}]
  \node {n}
    child {node {n-1}
      child {node {n-2}
      	child {node{...}
      		child {node{1}}
      		child {node{1}}
      		}
      	child {node{...}}
      }
      child {node {n-2}
      	child {node{...}}
      	child {node{...}}}
    }
    child {node {n-1}
    child {node {n-2}
      child {node{...}}
      child {node{...}}}
      child {node {n-2}
      child {node{...}}
      child {node{...}}}
    };
\end{tikzpicture}

Drzewo jest pełne i kończy się na liściu o wartości 1. Nietrudno zauważyć, że wysokosć drzewa to $n$.\\
A więc na $k$-tym poziomie drzewa koszt poziomu wynosi $2^k(n-k)$
W takim razie $T(n) = \sum_{k = 0}^{n} 2^k(n-k) = 2^{n+1} - n -2 = O(2^n)$. Niedobrze.

\subsection{Algorytm C}
$T(n) = 9T(\frac{n}{3}) + O(nlog(n^2))$\\

Ponownie możemy zastosować \textit{Twierdzenie o Rekurencji Uniwersalnej}.
U nas: 
\begin{align*}
	&a = 9\\
	&b = 3\\
	&\log_3 9 = 2\\
	&d = 2
\end{align*}
Bardzo porządny algorytm. $d = \log_3 9$, bo $2 = 2$ zatem $T(n) = O(n^2\log(n))$

\subsection{Wybór}
Algorytm B nie wchodzi w grę, jego złożoność obliczeniowa rośnie szybciej niż ceny alkoholu przed wprowadzeniem prohibicji.\\
Porównując algorytm A i C, ze złożonościami $O(n^{\log_2 5}) \approx O(n^{2.32})$ i $O(n^2\log(n))$ możemy skorzystać z tego, że dowolny wielomian rośnie szybciej niż logarytm innymi słowy $n^{2.32} > n^2log(n)$ już dla stosunkowo niedużych $n$-ów. Wybrałbym więc algorytm C.
\end{document}

\end{document}