\documentclass{article}
\usepackage{amsmath}
\usepackage[utf8]{inputenc} % `utf8` option to match Editor encoding
\usepackage[T1]{fontenc}
\usepackage[ruled,vlined]{algorithm2e}

\begin{document}
\section{Lista 2, Zadanie 8}
Algorytm ma przygotować tablicę B będącą tablicą sum prefiksowych. Zakładam, że algorytm następnie ma czekać na input i, j dany przynajmniej kilkakrotnie, w przeciwnym razie bezcelowe byłoby tworzenie dodadtkowej tablicy, wystraczyłoby policzyć sumę od A[i] do A[j] i zwrócić. Natomiast w przypadku gdy nasz algorytm ma działać na kształ funkcji generującej pamiętającej swój stan, jest sens generować tablicę B. 

\begin{algorithm}[H]
 \KwData{ tablica liczb A o długości N; indeksy i, j}
 \KwResult{ suma elementów tablicy A od i do j włącznie}
 \medskip
 B[1] = 0\\
 \For{ k from 1 to N}{
 	B[k+1] = B[k] + A[k]
 }
  \medskip
  read i,j\\
 \Return B[j+1] - B[i]
 \caption{Suma częściowa}
\end{algorithm}

\subsection{Złożoność czasowa}
FAZA I: Polegająca na generowaniu tablicy ma złożoność $T(n) = O(n)$, ponieważ pętla raz przechodzi po wszytskich elementach tablicy A.\\\\
Faza II: Polegająca na zwróceniu opowiedzi ma złożoność $T(n) = O(1)$ ponieważ są to już stałe operacje na wskażnikach i wartośćiach tablicy B.

\subsection{Rozmiar tablicy B}
Tablica B zajmuje $(N+1) \cdot sizeof(datatype(B))$. $N+1$ ponieważ ustawaimy B[1].

\subsection{Czas odpowiedzi}
Po wygenerowaniu tablicy B, czas odpowiedzi na pytanie jest $O(1)$.

\end{document}